\documentclass{book}
\usepackage{pgfplots}
\usepackage{imakeidx}
\usepackage{amsfonts}
\usepackage{amsthm}
\usepackage{tikz}

\newcommand{\SectionBreak}{%
    %\vskip 0.5ex

    \nointerlineskip
    \moveright 0.125\textwidth \vbox{\hrule width0.75\textwidth}
    \nointerlineskip
    %\vskip 0.5ex
    \makeatletter
        %\@afterindenfalse%
    \makeatother

}

\newtheorem{definition}{Definizione}

\pgfplotsset{compat=1.18}

\author{F. Piazza \and G. Michieletto}
\title{%
	Appunti di Analisi Matematica II \\
	\large corso della prof.ssa B.Noris \\
	Politecnico di Milano
}


\begin{document}


\begin{titlepage}
\maketitle

% \titlegraphic[]{
% \begin{tikzpicture}[remember picture]
% \begin{axis}
% \addplot3[surf, samples=50, opacity=0.5] {x^2+y^2};
% \end{axis}
%\end{tikzpicture}
% }

 

\end{titlepage}

\begin{chapter}{Equazioni differenziali}
\begin{section}{Equazioni differenziali del 1° ordine}
\begin{definition}
	Una equazione differenziale del 1° ordine è una relazione tra una funzione $y$ e la sua derivata $y'$ che può essere scritta come
	\begin{equation}
		y' = f(y)
	\end{equation}
	dove $f$ è una funzione continua su un intervallo $I$ di $\mathbb{R}$.
\end{definition}

Esempi:
\begin{itemize}
	\item $y' = t \sqrt{y_{(t^2)}+1}$ è in forma normale con $f(t,s) = t \sqrt{s^2+1}$. Il dominio di $f$ è $I = \mathbb{R} \times \mathbb{R}$.
	\item $y'_{(t)} = \frac{1}{t}$ con $t>0$ diventa $f(t,s) = \frac{1}{t}$. \textbf{Oss:} $f$ non dipende esplicitamente da $s$. \\ Il dominio di $f$ è $\{(t,s) \in \mathbb{R}^2 : s\in\mathbb{R}, t\in\mathbb{R}^* \}$, dunque è diviso in due parti. Dovrò quindi risolvere la EDO nelle due regioni.
\end{itemize}
\SectionBreak
\begin{definition}
	Si chiama \emph{integrale generale} l'insieme delle soluzioni.
\end{definition}
\begin{definition}
	Si chiama \emph{soluzione particolare} una specifica soluzione.
\end{definition}
\noindent{Una EDO del 1° ordine ha $\infty^1$, soluzioni, cioè avrà una costante arbitraria. In modo analogo, una EDO del 2° ordine avrà $\infty^2$ soluzioni, cioè avrà due costanti arbitrarie.}
Esempi:
\begin{itemize}
	\item integrale generale $ce^t$ con $c$ costante arbitraria. Esempi di soluzioni particolari: $e^t$, $2e^t$, $-e^t$.
	\item $z_{(t)} = -1 + arctan(t)$ con $t\in\mathbb{R}^*$. Esempio di soluzione: $z' = 0 + \frac{1}{1+t^2}$.
\end{itemize}
\textbf{Oss:} La EDO $y'_{(t)} = f(t,y_{(t)})$ è definita per $(t,y)\in dom(f)$
\end{section}

\end{chapter}

\end{document}